\documentclass[DIV15]{scrartcl}
% \usepackage[osf,sc]{mathpazo}
% \usepackage{eulervm}

\usepackage{lazyeqn}
\usepackage{booktabs}
\makeatletter
% \texttt{\textbackslash\expandafter\@gobble\string#1}
% \newcommand{\showcasemath}[1]{\expandafter\verb+#1+ & $#1$}
\makeatother


\makeatletter
\newcommand{\myverb}{%
    \begingroup
    % deactivate special characters
    \let\do\@makeother
    \dospecials
    % change '{' and '}' back to normal
    \catcode`\{=1
    \catcode`\}=2
    \@myverb%
}
\def\@myverb#1{%
    \endgroup%
    \texttt{#1}%
}
% \newcommand{\scm}[3]{%
%   % \vspace{2.1ex}
%   \noindent
%   \def\tmpp{#1}%
%   \def\tmp{#3}\fbox{%
%   \begin{minipage}{\linewidth}\vspace{0.5ex}
%   \begin{minipage}[t]{0.17\linewidth}
%     \texttt{\expandafter\foo\meaning\tmpp}
%   \end{minipage}
%   \begin{minipage}[t]{0.37\linewidth}%
%     #2
%   \end{minipage}
%   \begin{minipage}[t]{0.24\linewidth}%
%     \texttt{\expandafter\foo\meaning\tmp}
%   \end{minipage}
%   \begin{minipage}[t]{0.20\linewidth}
%     #3
%   \end{minipage}\vspace{0.5ex}
%   \end{minipage}}\newline\noindent
% }

\newcommand{\scm}[3]{%
  % \vspace{2.1ex}
  \noindent
  \def\tmpp{#1}%
  \def\tmp{#3}%
  \begin{minipage}{\linewidth}\vspace{0.5ex}
  \begin{minipage}[t]{0.17\linewidth}
    \texttt{\expandafter\foo\meaning\tmpp}
  \end{minipage}
  \begin{minipage}[t]{0.37\linewidth}%
    #2
  \end{minipage}
  \begin{minipage}[t]{0.24\linewidth}%
    \texttt{\expandafter\foo\meaning\tmp}
  \end{minipage}
  \begin{minipage}[t]{0.20\linewidth}
    #3
  \end{minipage}\vspace{0.5ex}
  \end{minipage}\newline\noindent
}

\makeatother
\def\foo#1>{}
\makeatletter
\newcommand{\showcasemath}{%
    \begingroup
    % deactivate special characters
    \let\do\@makeother
    \dospecials
    % change '{' and '}' back to normal
    \catcode`\{=1
    \catcode`\}=2
    \@myverb%
}
\makeatother

\title{\texttt{lazyeqn}\textnormal{ --- equation macros for \LaTeXe}}
\author{H.\ Onur Solmaz}
\begin{document}
\maketitle

% \section{Introduction}

% People who use \LaTeX{} everyday eventually realize that vanilla \LaTeX{} with
% amsmath and similar packages is not sufficiently rich with operators or
% symbols to enable abstraction that is necessary for typing fast---at least not
% fast enough. Most of these people realize that their typing involves a lot of
% repetition which can be reduced through abstraction. They define macros for
% symbols, operators, and anything that can be thought up which makes sense in the
% formulatory context.

% It cannot be expected from amsmath---which is the primary package for
% typesetting math---to include such \emph{contextual macros}. Its purpose is to
% constitute the robust and elegant basis for which the contextual macros are
% supposed to be built upon; and that it succeeds in. Nevertheless I have
% witnessed the acts of a lot of colleagues of mine who are not aware of such
% capabilities, and do everything the hard, vanilla way.

% Surely, independent of their disciplines, Physicists, Chemists, Engineers,
% Mathematicians, Computer Scientists etc. all use some sorts of mathematics.
% Specifically, they solve equation systems, differential equations, algebraic
% equations of different sorts, which all require utilization of different symbols
% in individual contexts. These disciplines often intersect, and can benefit from
% a common package which shall provide all the symbols required by respective
% contexts. \texttt{lazyeqn} aims to be that package. It is about standardization and
% comfort. It reduces the number of characters you are supposed to type by 50\%.

% Check out the examples section to see how using \texttt{lazyeqn} can simplify
% your life.
% \section{Examples}

% Let's start with a simple and minimalist example.
% Poisson's equation in Cartesian coordinates is written as
% %
% \begin{equation}
%   \left( \frac{\partial^2}{\partial x^2} + \frac{\partial^2}{\partial y^2} +
%   \frac{\partial^2}{\partial z^2} \right)\varphi(x,y,z) = f(x,y,z).
% \end{equation}
% %
% I took this example directly from Wikipedia, where the MediaWiki plugin
% \texttt{texvc} validates the equation using a weird parser written in OCaml,
% creates a \LaTeX{} file which contains only that equation and compiles it into a
% PNG. I explain all this, because there are no options to use any packages for
% equations in Wikipedia, and everything must be vanilla amsmath. As such, the
% equation above is typeset with the following code
% %
% \begin{verbatim}
% \begin{equation}
%   \left(\frac{\partial^2}{\partial x^2}+\frac{\partial^2}{\partial y^2}+
%   \frac{\partial^2}{\partial z^2}\right)\varphi(x,y,z)=f(x,y,z)
% \end{equation}
% \end{verbatim}
% %
% Now, with \texttt{lazyeqn}, it becomes
% %
% % \begin{equation}
% %   \rbr{\partdd{}{x}+\partdd{}{y} + \partdd{}{z}}\varphi(x,y,z) = f(x,y,z)
% % \end{equation}
% %
% \begin{verbatim}
% \begin{equation}
%   \rbr{\partdd{}{x}+\partdd{}{y}+\partdd{}{z}}\varphi(x,y,z)=f(x,y,z)
% \end{equation}
% \end{verbatim}
% %
% To compare, vanilla amsmath description has 134 characters, and with
% \texttt{lazyeqn} the number reduces to 72. That is almost a \textbf{50\% reduction}!

% Moving onto an example which uses letter symbols (like bold, calligraphic etc.).
% The symmetric fourth-order identity tensor is written as
% %
% \begin{equation}
%   \BFI^{(s)} =\half(\delta_{ik}~\delta_{jl}+\delta_{il}~\delta_{jk})~\basis{ijkl}
% \end{equation}
% %
% With vanilla amsmath, it is
% \begin{verbatim}
% \begin{equation}
%   \boldsymbol{\mathsf{I}}^{(s)}=\frac{1}{2}~(\delta_{ik}~\delta_{jl}+
%   \delta_{il}~\delta_{jk})~\mathbf{e}_i\otimes\mathbf{e}_j\otimes
%   \mathbf{e}_k\otimes\mathbf{e}_l
% \end{equation}
% \end{verbatim}
% %
% However with \texttt{lazyeqn} it becomes
% %
% \begin{verbatim}
% \begin{equation}
%   \BFI =\half(\delta_{ik}~\delta_{jl}+\delta_{il}~\delta_{jk})~\basis{ijkl}
% \end{equation}
% \end{verbatim}
% %
% The reduction ratio this time is 79 to 164, which is 51\% reduction in number of
% characters. It is arguable that time is vary valuable, but with
% \texttt{lazyeqn}, you can type twice as fast, without taking into account
% the additional time required to debug a code twice the size it should be.


\section{Operator Macros}

\subsection{Braces}

% \scm{\rbr}{Round braces}{$\rbr{A}$}
\scm{\rbr}{Round braces - adaptive}{$\rbr{A}$}
\scm{\rbrn}{Round braces - normal sized}{$\rbrn{A}$}
\scm{\rbrbig}{Round braces - big}{$\rbrbig{A}$}
\scm{\rbrBig}{Round braces - Big}{$\rbrBig{A}$}
\scm{\rbrbigg}{Round braces - bigg}{$\rbrbigg{A}$}
\scm{\rbrBigg}{Round braces - Bigg}{$\rbrBigg{A}$}
%
\scm{\sbr}{Square brackets - adaptive}{$\sbr{A}$}
\scm{\sbrn}{Square brackets - normal sized}{$\sbrn{A}$}
\scm{\sbrbig}{Square brackets - big}{$\sbrbig{A}$}
\scm{\sbrBig}{Square brackets - Big}{$\sbrBig{A}$}
\scm{\sbrbigg}{Square brackets - bigg}{$\sbrbigg{A}$}
\scm{\sbrBigg}{Square brackets - Bigg}{$\sbrBigg{A}$}
%
\scm{\cbr}{Curly braces - adaptive}{$\cbr{A}$}
\scm{\cbrn}{Curly braces - normal sized}{$\cbrn{A}$}
\scm{\cbrbig}{Curly braces - big}{$\cbrbig{A}$}
\scm{\cbrBig}{Curly braces - Big}{$\cbrBig{A}$}
\scm{\cbrbigg}{Curly braces - bigg}{$\cbrbigg{A}$}
\scm{\cbrBigg}{Curly braces - Bigg}{$\cbrBigg{A}$}
%
% \scm{\abr}{Angles}{$\abr{A}$}
\scm{\abr}{Angles - adaptive}{$\abr{A}$}
\scm{\abrn}{Angles - normal sized}{$\abrn{A}$}
\scm{\abrbig}{Angles - big}{$\abrbig{A}$}
\scm{\abrBig}{Angles - Big}{$\abrBig{A}$}
\scm{\abrbigg}{Angles - bigg}{$\abrbigg{A}$}
\scm{\abrBigg}{Angles - Bigg}{$\abrBigg{A}$}

\scm{\dbr}{Double square braces - adaptive}{$\dbr{A}$} % maybe use rrbracket-llbracket from stmaryd?
\scm{\dbrl}{Left double square bracket}{$\dbrl$}
\scm{\dbrr}{Right double square bracket}{$\dbrr$}
%
\subsection{Operators}

% \begin{table}[H]
%   \centering
%   \begin{tabular}{l l l l}
%     \toprule
%     Macro & Explanation & Example usage & Result \\
%     \midrule
%     \verb+\tra+ & Transpose & &  \\
%     \verb+\variation{}{}+ & Variation & \verb+\variation{a}{b}+ & $\variation{a}{b}$  \\
%     \verb+\partd{}{}+ & Partial derivative & & \\
%     \verb+\partdb{}{}+ & Partial derivative with brackets & & \\
%     \verb+\cpartd{}{}+ & Partial derivative using \verb+\cfrac+ & & \\
%     \bottomrule
%   \end{tabular}
% \end{table}

% \scm{Macro}{Explanation}{Example usage}
%
\scm{\transposesymbol}{Transpose symbol}{$\transposesymbol$}
\scm{\tra}{Transpose}{$ A\tra$}
% \scm{\tra[]}{Transpose with preceding argument}{$\tra[-]$}
\scm{\trab}{Transpose with preceding argument}{$ A\trab{-}$}
\scm{\traf}{Transpose with following argument}{$ A\traf{-1}$}
\scm{\trabf}{Transpose with preceding and following argument}{$ A\trabf{-}{-1}$}
\scm{\trat}{Transpose with argument on top}{$ A\trat{23}$}
\scm{\inv}{Inverse}{$ A\inv$}
\scm{\invtra}{Invert, transpose}{$ A\invtra$}
\scm{\trainv}{Transpose, invert}{$ A\trainv$}
\scm{\ovd}{Time derivative, over-dot}{$\ovd{a}$}
\scm{\ovd}{Double over-dot}{$\ovdd{a}$}
\scm{\ovd}{Spatial derivative, over-prime}{$\ovp{a}$}
\scm{\ovd}{Double over-prime}{$\ovpp{a}$}
\scm{\dif}{Differential operator d}{$\dif a$}
\scm{\del}{Partial derivative operator}{$\del a$}
\scm{\var}{Variational operator d}{$\var a$}
\scm{\vartn}{Variation}{$\vartn{a}{b}$}
\scm{\partd}{Partial derivative}{$\partd{a}{b}$}
\scm{\partdd}{Double partial derivative}{$\partdd{a}{b}$}
\scm{\epartd}{Euler type partial derivative}{$\epartd{a}{b}$}
\scm{\deriv}{Derivative}{$\deriv{a}{b}$}
\scm{\dderiv}{Double derivative}{$\dderiv{a}{b}$}
\scm{\Deriv}{Material derivative}{$\Deriv{a}{b}$}
\scm{\DDeriv}{Double material derivative}{$\DDeriv{a}{b}$}
\scm{\ederiv}{Euler type derivative}{$\ederiv{a}{b}$}
%
\scm{\ddt}{Time derivative}{$\ddt x$}
\scm{\ddelt}{Partial Time derivative}{$\ddelt x$}
%
\scm{\ddx}{Arbitrary derivative}{$\ddx a$}
\scm{\ddelx}{Arbitrary partial derivative}{$\ddelx a$}
%
\scm{\assem}{Assembly symbol}{$\assem$}
\scm{\evat}{Evaluated at}{$a\evat_{b=0}$}
\scm{\degree}{Degree symbol}{\degree{}C}
\scm{\suml}{Summation made easier, with limits}{$\suml{i=1}{3}$}
\scm{\sumn}{Summation made easier, with nolimits}{$\sumn{i=1}{3}$}
\scm{\dtp}{Dot product (\texttt{cdot} shortcut)}{$ A\dtp B$}
\scm{\crs}{Cross product (\texttt{times} shortcut)}{$ A\crs B$}
% \scm{\dcrs}{Double cross product }{$ A\dcrs B$}
\scm{\dyd}{Dyadic product (\texttt{otimes} shortcut)}{$ A\dyd B$}
\scm{\linedot}{Time integral with overline}{$\linedot{A}$}
\scm{\DIV}{DIVergence}{$\DIV A$}
\scm{\Div}{Divergence}{$\Div A$}
\scm{\div}{divergence}{$\div A$}
\scm{\curl}{Curl}{$\curl A$}
\scm{\rot}{Rotation}{$\rot A$}
\scm{\proj}{Projection}{$\proj A$}
\scm{\sign}{Sign function}{$\sign A$}
\scm{\unit}{Unit function}{$\unit A$}
\scm{\arg}{arg}{$\arg A$}
\scm{\Arg}{Arg}{$\Arg A$}
\scm{\sym}{Symmetric}{$\sym A$}
\scm{\TR}{TRace}{$\TR A$}
\scm{\tr}{trace}{$\tr A$}
\scm{\trial}{Trial function}{$\trial A$}
\scm{\DEV}{DEViatoric}{$\DEV A$}
\scm{\dev}{deviatoric}{$\dev A$}
\scm{\GRAD}{GRADient}{$\GRAD A$}
\scm{\Grad}{Gradient}{$\Grad A$}
\scm{\grad}{gradient}{$\grad A$}
\scm{\Lin}{Linearization}{$\Lin A$}
\scm{\lin}{linearization}{$\lin A$}
\scm{\skw}{Skew}{$\skw A$}
\scm{\loc}{loc}{$\loc A$}
\scm{\cof}{Cofactor}{$\cof A$}
\scm{\reac}{reac}{$\reac A$}
\scm{\pind}{pind}{$\pind A$}
\scm{\ndim}{ndim}{$\ndim$}
\scm{\diag}{diag}{$\diag A$}
\scm{\veci{}}{Indexed list representation}{$\veci{a}$}
\scm{\dV}{Volume element}{$\dV$}
\scm{\dv}{Volume element}{$\dv$}
\scm{\dA}{Area element}{$\dA$}
\scm{\da}{Area element}{$\da$}
\scm{\dS}{Surface element}{$\dS$}
\scm{\dx}{Line element}{$\dx$}
\scm{\dt}{Time difference}{$\dt$}
\scm{\Abs{}}{Absolute value}{$\Abs{A}$}
\scm{\Norm{}}{Norm}{$\Norm{A}$}
\scm{\basis{}}{Vector bases}{$\basis{ijkl}$}
\scm{\rbdot}{Operands}{$\div\rbdot$}
\scm{\so}{Special orthogonal group}{$\so$}
% \scm{\rve}{}{$\rve$}
% \scm{\rpg}{}{$\rpg$}
% \scm{\ehs}{}{$\ehs$}
% \scm{\mo}{}{$\mo$}
\scm{\othree}{Orthogonal group 3}{$\othree$}
\scm{\sothree}{Special orthogonal group 3}{$\sothree$}
\scm{\slthree}{Special linear group 3}{$\slthree$}
\scm{\eqand}{and between equations}{$a \eqand b$}
\scm{\eqor}{and between equations}{$a \eqor b$}
\scm{\eqwith}{with between equations}{$a \eqwith b$}
\scm{\eqcomma}{Comma between equations}{$a\eqcomma b$}
\scm{\eqsemi}{Semicolon between equations}{$a\eqsemi b$}
\scm{\quadd}{Wraps the parameters with quads}{$a\quadd{\rightarrow} b$}

\subsection{Fractions, roots, roman numerals}
\scm{\half}{Half}{$\half$}
\scm{\third}{Third}{$\third$}
\scm{\fourth}{Fourth}{$\fourth$}
\scm{\fifth}{Fifth}{$\fifth$}
\scm{\sixth}{Sixth}{$\sixth$}
\scm{\seventh}{Seventh}{$\seventh$}
\scm{\eighth}{Eighth}{$\eighth$}
\scm{\ninth}{Ninth}{$\ninth$}
\scm{\tenth}{Tenth}{$\tenth$}
\scm{\twothree}{Twothree}{$\twothree$}
\scm{\threetwo}{Threetwo}{$\threetwo$}
\scm{\sqonetwo}{Sqonetwo}{$\sqonetwo$}
\scm{\sqonethree}{Sqonethree}{$\sqonethree$}
\scm{\sqtwothree}{Sqtwothree}{$\sqtwothree$}
\scm{\sqthreetwo}{Sqthreetwo}{$\sqthreetwo$}
\scm{\Rone}{Roman 1}{$\Rone$}
\scm{\Rtwo}{Roman 2}{$\Rtwo$}
\scm{\Rthree}{Roman 3}{$\Rthree$}
\scm{\Rfour}{Roman 4}{$\Rfour$}
\scm{\Rvarfour}{Roman var 4}{$\Rvarfour$}
\scm{\Rfive}{Roman 5}{$\Rfive$}
\scm{\Rsix}{Roman 6}{$\Rsix$}
\scm{\Rseven}{Roman 7}{$\Rseven$}
\scm{\Reight}{Roman 8}{$\Reight$}
\scm{\Rnine}{Roman 9}{$\Rnine$}
\scm{\Rten}{Roman 10}{$\Rten$}

\subsection{Vectors}
\scm{\cvec{}}{Column vector}{$\cvec{1;2;3;4}$}
\scm{\vcvec{}}{Column vector with straight braces}{$\vcvec{1;2;3;4}$}
\scm{\rvec{}}{Row vector}{$\rvec{1;2;3;4}$}
\scm{\vrvec{}}{Row vector with straight braces}{$\vrvec{1;2;3;4}$}

\end{document}
